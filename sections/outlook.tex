
\section*{Outlook}

We end this article by outlining possible future directions for enclaves and dedicated security chips. We also discuss how these two closely-related but still different hardware-assisted security mechanisms can work in collaboration.

Future computing architectures will, most likely, be more diverse than the ones we use today. The paradigm that has been prevalent for the past few decades is to perform most of the computation in one or more general-purpose processing units. Modern computing platforms often incorporate more specialized computing units like GPUs. Future platforms are likely to combine a more diverse set of computing units including CPUs, GPUs, TPUs, FPGAs and more. %~\cite{dean2018new}. 
Similar to the currently popular enclave architectures that are implemented in the main CPUs of the computing platforms, also other processing units like GPUs and FPGAs will need Trusted Execution Environments (TEEs). There are already several on-going research efforts that explore the design of TEEs for such computing units~\cite{volos2018graviton}.

Besides secure computation, such various computing units should also be able to communicate securely with each other. Our research on trusted path highlights that also peripherals like I/O devices need secure communication with enclaves and TEEs. Furthermore, the need for secure communication between peripherals and TEEs is not limited to IO devices alone, but also other peripherals like GPS units and fingerprint sensors should be able to communicate securely with enclaves. 

Such secure communication between TEEs and other trusted platform components requires secure authentication, enclave identification and access control mechanism. The ARM TrustZone architecture that is used in many mobile devices has limited support towards this direction. In TrustZone, the communication bus that connects the different on-chip components like memory controllers carries a flag that indicates whether the main CPU runs in secure world or normal world mode. This allows other hardware components to make coarse-grained access control decisions based on the CPU's execution mode~\cite{ekberg2014untapped}. However, this approach lacks the possibility of distinguishing one enclave from another. Such intra-platform communication is also not secure against simple physical attacks like bus tapping. 

Extending this paradigm for more fine-grained access control and strongly-secured inter-component communication is one promising future direction for future research. This approach could be used to build future computing platforms where various enclaves, peripherals and special-purpose security chips all work together to provide a rich set of platform security services.
