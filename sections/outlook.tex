
\section*{Outlook}

%We end this article by outlining possible future directions for enclaves and dedicated security chips. We also discuss how these two closely-related but still somewhat different hardware-assisted security mechanisms can work in collaboration.

%Future computing architectures are likely to be more diverse than the ones we use today. The paradigm that has been prevalent for the past few decades is to perform most of the computation in one or more general-purpose processing units. Modern computing platforms often incorporate more specialized computing units like GPUs. 

Future computing platforms are likely to combine various computing units like CPUs, GPUs, TPUs, FPGAs and more. %~\cite{dean2018new}.
Similar to the current enclave architectures that enhance CPUs with secure execution capabilities, also other processing units like GPUs and FPGAs will need secure computation. There are already several on-going research efforts that explore the design of such TEEs~\cite{volos2018graviton}.

Our research on trusted path (the \protection system~\cite{protection}) highlights that I/O devices need secure communication with enclaves. Similarly, also other peripherals like GPS units and fingerprint sensors would benefit from secure communication with enclaves. Protected communication between TEEs and other platform components requires authentication, enclave identification and access control mechanisms. The ARM TrustZone architecture has limited support towards this direction. In TrustZone, hardware components like memory controllers can make coarse-grained access control decisions based on the CPU's execution mode~\cite{ekberg2014untapped}. 

%However, this approach lacks the possibility of distinguishing one enclave from another. Such intra-platform communication is also not secure against simple physical attacks like bus tapping. 

Extending this paradigm for more fine-grained access control and secure inter-component communication is one promising direction. We envision future computing platforms where enclaves, peripherals and special-purpose security chips can communicate and work together to provide a rich set of hardware-assisted platform security services.
