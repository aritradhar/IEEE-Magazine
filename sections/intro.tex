
%\section*{}
\vspace{10pt}

Trusted Computing Base (TCB) minimization is one of the most fundamental computer security principles. The main idea is to reduce the amount of software and hardware that needs to be trusted for the secure operation of a particular application. A common technique to achieve TCB minimization is to run the application inside a Trusted Execution Environment (TEE). The TEE protects the application's execution, despite any other compromised software on the same system.

One TEE implementation approach that has gained significant popularity recently is to realize the TEE by enhancing the main CPU of the computing platform with new features like special instructions and access control checks. Intel's SGX, designed for the x86 architecture, is a prime example of such TEE. In SGX, the CPU ensures that no other process can access the memory of the protected application that is called an \emph{enclave}. By doing this, SGX guarantees that enclaves enjoy execution integrity, and their data remains confidential.  

Several other TEE designs exist too. ARM TrustZone is a popular TEE architecture that is used in many commercial mobile devices, while Sanctum~\cite{sanctum} serves as a good example of a research TEE system. For simplicity, we focus on Intel's SGX and use it as a case study to discuss the strengths and limitations of enclaves.

SGX-style enclaves are powerful security primitive. They are \emph{programmable}, and thus developers can implement almost arbitrary hardware-protected security services using them. This is in contrast to previous secure elements like TPMs that support only a fixed set of operations. Enclaves are also \emph{fast}, as they run on the main CPU of the computing platform, compared to significantly slower security elements like smart cards. And furthermore, enclaves are \emph{cheap}, since they require no additional hardware in contrast to expensive separate co-processors like HSMs. 

This combination of programmability, high performance, and low cost makes enclaves an attractive way to deploy various hardware-assisted security services. Indeed, after a decade of research and development into secure enclaves, the first large-scale commercial deployments are now starting. For example, Microsoft's Confidential Computing service uses SGX enclaves to protect customer data in the cloud.

The wide adoption of enclave architectures in modern CPUs is probably the most prominent trend in hardware-assisted security over the last decade. However, there is also another, more subtle trend appearing. Recently, computing service providers like Google and computer manufacturers like Apple have started to enhance their systems with special-purpose security chips. Google's cloud servers have a security chip called Titan in them~\cite{titan}, while Apple's computers come with the T2 security chip~\cite{t2}. 

At first glance, these two trends seem almost contradictory. If enclaves enable arbitrary hardware-protected security services, why do we still need dedicated security chips? 

In this article, we discuss the rationale behind this trend. We explain the benefits of dedicated security chips and outline two of our research projects where we designed such. These projects showcase an interesting new pattern --- one where special-purpose security chips assist enclaves and thus improve their security.
