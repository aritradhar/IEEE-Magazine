
\section*{Comparison}

\begin{figure}[t]
	\centering
	\includegraphics[scale=0.5]{comparison.pdf}
	\caption{Comparison of \protection and \proximitee with respect to the platform-integrated security chips, user-pluggable security tokens and CPU-0implemented enclave architecture.}
\label{fig:prototype}   
\end{figure}


Now that we have seen two examples of commercially-deployed, integrated security chips (Titan and T2) and another two examples of research projects where we designed user-pluggable security tokens (\protection and \proximitee), we can compare these solutions. 

\subsubsection*{Deployment}

As the first comparison dimension we consider deployment. Integrated security chips like Titan or T2 can be only issued by major service and platform providers that have the possibility to design and build their own systems. \proximitee is an example of a plug-and-play security token that can be attached to the target platform over a standard interface like USB. Obviously, the deployment of security solutions is a feasible option for a larger set of service providers. For example, small and medium cloud computing providers can enhance off-the-shelf servers with \proximitee tokens to enable secure, replay-protected attestation on them. 

Also \protection could be deployed as a plug-and-play security token. User-pluggable token is a good deployment option, for example, for web-based online services like e-voting and online banking. Such deployment allows voting authorities and banks to increase the security of their services without restricting the users’ choice of client platform. In medical and industrial domains, an externally-attached \protection module can improve the security of safety-critical systems, even when modifications to the computing platform itself are prohibited due to strict regulations. Alternatively, \protection could be deployed as an integrated security chip such that its functionality is implemented as part of the integrated keyboard, mouse and display controllers of the computing platform. 


\subsubsection*{Security Benefits}

As the next comparison dimension, we discuss the main security benefit of dedicated security chips. Titan and T2 are chips that primarily enable functionality like secure boot that is missing from enclaves. Thus, the operation of Titan and T2 is largely orthogonal to the operation of enclaves. 

In comparison, \proximitee is a security chip that is designed to work in collaboration with enclaves and improve their security guarantees by enabling secure and replay-protected TEE identification for remote attestation. 

\protection is a solution that can either assist enclaves or operate independently of them. One possible usage for \protection is to enable a trusted path from the user to a local enclave, which can in turn can communicate secure with remote servers. In such a deployment, \protection works in collaboration with an enclave and addresses one of the limitations of the SGX architecture -- it’s lack of trusted path. Alternatively, \protection could be used to create a trusted path from the user to a remote server without the use of enclaves. Such deployment option may be beneficial, for example, in cases where the risk of microarchitectural attacks on enclaves is considered too high and thus the integrity of the trusted path should not rely on enclave security.


